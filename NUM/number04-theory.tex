

\chapter{\protect\hyperlink{chap:\thechapter}{Number Theor
}
\addtocontents{toc}{\protect\hypertarget{chap:\thechapter}{}}








\chapter{Number Theory}
\section{recognise the tests of divisibility of 2, 3, 5
and 10}



\section{recognise the tests of divisibility of 4, 6, 8 and 9}


\section{Perform prime factorisation of positive integers}
Students are required to use prime
factorisation and short division to find the
greatest common divisor and the least
common multiple.


\section{Multiples}
We know that $12$ equals $3$ times $4$. In other words, $12$ equals some integer times $4$. For that reason, we say that $12$ is a \textbf{multiple}  of 4.
\begin{definition}
Let $\color[rgb]{0.11,0.21,0.37}a$ and $\color[rgb]{0.11,0.21,0.37}b$ be numbers. We say that $\color[rgb]{0.11,0.21,0.37}a$ is a multiple of $\color[rgb]{0.11,0.21,0.37}b$ if $\color[rgb]{0.11,0.21,0.37}a$ equals $\color[rgb]{0.11,0.21,0.37}b$ times some integer. In other words, $\color[rgb]{0.11,0.21,0.37}a$ is a multiple of $\color[rgb]{0.11,0.21,0.37}b$ if there is an integer $\color[rgb]{0.11,0.21,0.37}n$ such that $\color[rgb]{0.11,0.21,0.37}a = bn$.
\end{definition}
For instance, 7 is not a multiple of 4, because we cannot write 7 as the  \textbf{product}  of 4 and an integer. Note that $-12$ is a multiple of $4$, because $-12$ equals $-3$ times $4$. Similarly, $0$ is a multiple of $4$, because $0$ equals $0$ times $4$.

In this chapter, we’ll talk about division differently than we did in Chapter 1. We’ll use the “quotient and remainder” concept of division, which you probably used when you first learned about division. As an example, when we divide 13 by 4, the quotient is 3 and the remainder is 1.

Using this view of division, we can say that an integer $a$ is a multiple of an integer $b$ if $a$ divided by $b$ has remainder 0. So, for example, 12 is a multiple of 4 because $12$ divided by $4$ has remainder 0, while 13 is not a multiple of 4 because $13$ divided by $4$ has remainder 1.
\section{find the greatest common divisor and the least
common multiple}

At Key Stage 2, students are required to find
the greatest common divisor and the least
common multiple of two numbers by listing
their multiples and factors, and using short
division.

\section{find the greatest common divisor and the least
common multiple}

At Key Stage 2, students are required to find
the greatest common divisor and the least
common multiple of two numbers by listing
their multiples and factors, and using short
division.



The terms “H.C.F.”, “gcd”, etc. can be used.







\section{}








\section{}

















\section{}













